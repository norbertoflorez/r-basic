% Options for packages loaded elsewhere
\PassOptionsToPackage{unicode}{hyperref}
\PassOptionsToPackage{hyphens}{url}
%
\documentclass[
]{article}
\usepackage{amsmath,amssymb}
\usepackage{lmodern}
\usepackage{ifxetex,ifluatex}
\ifnum 0\ifxetex 1\fi\ifluatex 1\fi=0 % if pdftex
  \usepackage[T1]{fontenc}
  \usepackage[utf8]{inputenc}
  \usepackage{textcomp} % provide euro and other symbols
\else % if luatex or xetex
  \usepackage{unicode-math}
  \defaultfontfeatures{Scale=MatchLowercase}
  \defaultfontfeatures[\rmfamily]{Ligatures=TeX,Scale=1}
\fi
% Use upquote if available, for straight quotes in verbatim environments
\IfFileExists{upquote.sty}{\usepackage{upquote}}{}
\IfFileExists{microtype.sty}{% use microtype if available
  \usepackage[]{microtype}
  \UseMicrotypeSet[protrusion]{basicmath} % disable protrusion for tt fonts
}{}
\makeatletter
\@ifundefined{KOMAClassName}{% if non-KOMA class
  \IfFileExists{parskip.sty}{%
    \usepackage{parskip}
  }{% else
    \setlength{\parindent}{0pt}
    \setlength{\parskip}{6pt plus 2pt minus 1pt}}
}{% if KOMA class
  \KOMAoptions{parskip=half}}
\makeatother
\usepackage{xcolor}
\IfFileExists{xurl.sty}{\usepackage{xurl}}{} % add URL line breaks if available
\IfFileExists{bookmark.sty}{\usepackage{bookmark}}{\usepackage{hyperref}}
\hypersetup{
  pdftitle={Tarea4},
  pdfauthor={Norberto Florez},
  hidelinks,
  pdfcreator={LaTeX via pandoc}}
\urlstyle{same} % disable monospaced font for URLs
\usepackage[margin=1in]{geometry}
\usepackage{color}
\usepackage{fancyvrb}
\newcommand{\VerbBar}{|}
\newcommand{\VERB}{\Verb[commandchars=\\\{\}]}
\DefineVerbatimEnvironment{Highlighting}{Verbatim}{commandchars=\\\{\}}
% Add ',fontsize=\small' for more characters per line
\usepackage{framed}
\definecolor{shadecolor}{RGB}{248,248,248}
\newenvironment{Shaded}{\begin{snugshade}}{\end{snugshade}}
\newcommand{\AlertTok}[1]{\textcolor[rgb]{0.94,0.16,0.16}{#1}}
\newcommand{\AnnotationTok}[1]{\textcolor[rgb]{0.56,0.35,0.01}{\textbf{\textit{#1}}}}
\newcommand{\AttributeTok}[1]{\textcolor[rgb]{0.77,0.63,0.00}{#1}}
\newcommand{\BaseNTok}[1]{\textcolor[rgb]{0.00,0.00,0.81}{#1}}
\newcommand{\BuiltInTok}[1]{#1}
\newcommand{\CharTok}[1]{\textcolor[rgb]{0.31,0.60,0.02}{#1}}
\newcommand{\CommentTok}[1]{\textcolor[rgb]{0.56,0.35,0.01}{\textit{#1}}}
\newcommand{\CommentVarTok}[1]{\textcolor[rgb]{0.56,0.35,0.01}{\textbf{\textit{#1}}}}
\newcommand{\ConstantTok}[1]{\textcolor[rgb]{0.00,0.00,0.00}{#1}}
\newcommand{\ControlFlowTok}[1]{\textcolor[rgb]{0.13,0.29,0.53}{\textbf{#1}}}
\newcommand{\DataTypeTok}[1]{\textcolor[rgb]{0.13,0.29,0.53}{#1}}
\newcommand{\DecValTok}[1]{\textcolor[rgb]{0.00,0.00,0.81}{#1}}
\newcommand{\DocumentationTok}[1]{\textcolor[rgb]{0.56,0.35,0.01}{\textbf{\textit{#1}}}}
\newcommand{\ErrorTok}[1]{\textcolor[rgb]{0.64,0.00,0.00}{\textbf{#1}}}
\newcommand{\ExtensionTok}[1]{#1}
\newcommand{\FloatTok}[1]{\textcolor[rgb]{0.00,0.00,0.81}{#1}}
\newcommand{\FunctionTok}[1]{\textcolor[rgb]{0.00,0.00,0.00}{#1}}
\newcommand{\ImportTok}[1]{#1}
\newcommand{\InformationTok}[1]{\textcolor[rgb]{0.56,0.35,0.01}{\textbf{\textit{#1}}}}
\newcommand{\KeywordTok}[1]{\textcolor[rgb]{0.13,0.29,0.53}{\textbf{#1}}}
\newcommand{\NormalTok}[1]{#1}
\newcommand{\OperatorTok}[1]{\textcolor[rgb]{0.81,0.36,0.00}{\textbf{#1}}}
\newcommand{\OtherTok}[1]{\textcolor[rgb]{0.56,0.35,0.01}{#1}}
\newcommand{\PreprocessorTok}[1]{\textcolor[rgb]{0.56,0.35,0.01}{\textit{#1}}}
\newcommand{\RegionMarkerTok}[1]{#1}
\newcommand{\SpecialCharTok}[1]{\textcolor[rgb]{0.00,0.00,0.00}{#1}}
\newcommand{\SpecialStringTok}[1]{\textcolor[rgb]{0.31,0.60,0.02}{#1}}
\newcommand{\StringTok}[1]{\textcolor[rgb]{0.31,0.60,0.02}{#1}}
\newcommand{\VariableTok}[1]{\textcolor[rgb]{0.00,0.00,0.00}{#1}}
\newcommand{\VerbatimStringTok}[1]{\textcolor[rgb]{0.31,0.60,0.02}{#1}}
\newcommand{\WarningTok}[1]{\textcolor[rgb]{0.56,0.35,0.01}{\textbf{\textit{#1}}}}
\usepackage{graphicx}
\makeatletter
\def\maxwidth{\ifdim\Gin@nat@width>\linewidth\linewidth\else\Gin@nat@width\fi}
\def\maxheight{\ifdim\Gin@nat@height>\textheight\textheight\else\Gin@nat@height\fi}
\makeatother
% Scale images if necessary, so that they will not overflow the page
% margins by default, and it is still possible to overwrite the defaults
% using explicit options in \includegraphics[width, height, ...]{}
\setkeys{Gin}{width=\maxwidth,height=\maxheight,keepaspectratio}
% Set default figure placement to htbp
\makeatletter
\def\fps@figure{htbp}
\makeatother
\setlength{\emergencystretch}{3em} % prevent overfull lines
\providecommand{\tightlist}{%
  \setlength{\itemsep}{0pt}\setlength{\parskip}{0pt}}
\setcounter{secnumdepth}{-\maxdimen} % remove section numbering
\ifluatex
  \usepackage{selnolig}  % disable illegal ligatures
\fi

\title{Tarea4}
\author{Norberto Florez}
\date{28/4/2021}

\begin{document}
\maketitle

\hypertarget{punto-1---realizad-los-siguientes-productos-de-matrices-siguiente-en-r}{%
\paragraph{Punto 1 - Realizad los siguientes productos de matrices
siguiente en
R:}\label{punto-1---realizad-los-siguientes-productos-de-matrices-siguiente-en-r}}

\[ A = \begin{equation*}\begin{pmatrix} 1 & 2 & 3 & 4\\ 4 & 3 & 2 & 1\\ 0 & 1 & 0 & 2 
\\ 3 & 0 & 4 & 0\end{pmatrix}\end{equation*}\]

\[ B = \begin{equation*}\begin{pmatrix} 4 & 3 & 2 & 1\\ 0 & 3 & 0 & 4
\\ 1 & 2 & 3 & 4\\ 0 & 1 & 0 & 2\end{pmatrix}\end{equation*}\]

\begin{Shaded}
\begin{Highlighting}[]
\NormalTok{A }\OtherTok{=} \FunctionTok{rbind}\NormalTok{(}\DecValTok{1}\SpecialCharTok{:}\DecValTok{4}\NormalTok{,}\DecValTok{4}\SpecialCharTok{:}\DecValTok{1}\NormalTok{,}\FunctionTok{c}\NormalTok{(}\DecValTok{0}\NormalTok{,}\DecValTok{1}\NormalTok{,}\DecValTok{0}\NormalTok{,}\DecValTok{2}\NormalTok{), }\FunctionTok{c}\NormalTok{(}\DecValTok{3}\NormalTok{,}\DecValTok{0}\NormalTok{,}\DecValTok{4}\NormalTok{,}\DecValTok{0}\NormalTok{))}
\NormalTok{A}
\end{Highlighting}
\end{Shaded}

\begin{verbatim}
##      [,1] [,2] [,3] [,4]
## [1,]    1    2    3    4
## [2,]    4    3    2    1
## [3,]    0    1    0    2
## [4,]    3    0    4    0
\end{verbatim}

\begin{Shaded}
\begin{Highlighting}[]
\NormalTok{B }\OtherTok{=} \FunctionTok{rbind}\NormalTok{(}\DecValTok{4}\SpecialCharTok{:}\DecValTok{1}\NormalTok{,}\FunctionTok{c}\NormalTok{(}\DecValTok{0}\NormalTok{,}\DecValTok{3}\NormalTok{,}\DecValTok{0}\NormalTok{,}\DecValTok{4}\NormalTok{),}\DecValTok{1}\SpecialCharTok{:}\DecValTok{4}\NormalTok{, }\FunctionTok{c}\NormalTok{(}\DecValTok{0}\NormalTok{,}\DecValTok{1}\NormalTok{,}\DecValTok{0}\NormalTok{,}\DecValTok{2}\NormalTok{))}
\NormalTok{B}
\end{Highlighting}
\end{Shaded}

\begin{verbatim}
##      [,1] [,2] [,3] [,4]
## [1,]    4    3    2    1
## [2,]    0    3    0    4
## [3,]    1    2    3    4
## [4,]    0    1    0    2
\end{verbatim}

\begin{Shaded}
\begin{Highlighting}[]
\NormalTok{t }\OtherTok{=} \DecValTok{4}
\end{Highlighting}
\end{Shaded}

\[A \cdot B\]

\begin{Shaded}
\begin{Highlighting}[]
\NormalTok{A}\SpecialCharTok{\%*\%}\NormalTok{B}
\end{Highlighting}
\end{Shaded}

\begin{verbatim}
##      [,1] [,2] [,3] [,4]
## [1,]    7   19   11   29
## [2,]   18   26   14   26
## [3,]    0    5    0    8
## [4,]   16   17   18   19
\end{verbatim}

\[B \cdot A\]

\begin{Shaded}
\begin{Highlighting}[]
\NormalTok{B}\SpecialCharTok{\%*\%}\NormalTok{A}
\end{Highlighting}
\end{Shaded}

\begin{verbatim}
##      [,1] [,2] [,3] [,4]
## [1,]   19   19   22   23
## [2,]   24    9   22    3
## [3,]   21   11   23   12
## [4,]   10    3   10    1
\end{verbatim}

\[(A \cdot B)^{t}\]

\begin{Shaded}
\begin{Highlighting}[]
\NormalTok{(A}\SpecialCharTok{\%*\%}\NormalTok{B)}\SpecialCharTok{\^{}}\NormalTok{t}
\end{Highlighting}
\end{Shaded}

\begin{verbatim}
##        [,1]   [,2]   [,3]   [,4]
## [1,]   2401 130321  14641 707281
## [2,] 104976 456976  38416 456976
## [3,]      0    625      0   4096
## [4,]  65536  83521 104976 130321
\end{verbatim}

\[B^{t} \cdot A\]

\begin{Shaded}
\begin{Highlighting}[]
\NormalTok{(B}\SpecialCharTok{\^{}}\NormalTok{t)}\SpecialCharTok{\%*\%}\NormalTok{A}
\end{Highlighting}
\end{Shaded}

\begin{verbatim}
##      [,1] [,2] [,3] [,4]
## [1,]  583  771  934 1137
## [2,] 1092  243 1186   81
## [3,]  833  131 1059  182
## [4,]   52    3   66    1
\end{verbatim}

\[(A \cdot B)^{-1}\]

\begin{Shaded}
\begin{Highlighting}[]
\NormalTok{(A}\SpecialCharTok{\%*\%}\NormalTok{B)}\SpecialCharTok{\^{}}\NormalTok{(}\SpecialCharTok{{-}}\DecValTok{1}\NormalTok{)}
\end{Highlighting}
\end{Shaded}

\begin{verbatim}
##            [,1]       [,2]       [,3]       [,4]
## [1,] 0.14285714 0.05263158 0.09090909 0.03448276
## [2,] 0.05555556 0.03846154 0.07142857 0.03846154
## [3,]        Inf 0.20000000        Inf 0.12500000
## [4,] 0.06250000 0.05882353 0.05555556 0.05263158
\end{verbatim}

\[A^{-1} \cdot B^{t}\]

\begin{Shaded}
\begin{Highlighting}[]
\NormalTok{(A}\SpecialCharTok{\^{}}\NormalTok{(}\SpecialCharTok{{-}}\DecValTok{1}\NormalTok{))}\SpecialCharTok{\%*\%}\NormalTok{(B}\SpecialCharTok{\^{}}\NormalTok{t)}
\end{Highlighting}
\end{Shaded}

\begin{verbatim}
##          [,1]     [,2] [,3]     [,4]
## [1,] 256.3333 127.0833 43.0 218.3333
## [2,]  64.5000  56.2500 44.5 229.5833
## [3,]      Inf      Inf  Inf      Inf
## [4,]      NaN      Inf  NaN      Inf
\end{verbatim}

\hypertarget{punto-2---considerad-en-un-vector-los-nuxfameros-de-vuestro-dni-y-llamadlo-dni.}{%
\paragraph{Punto 2 - Considerad en un vector los números de vuestro DNI
y llamadlo
dni.}\label{punto-2---considerad-en-un-vector-los-nuxfameros-de-vuestro-dni-y-llamadlo-dni.}}

\[dni = (9, 4, 1, 5, 1, 0, 0, 9)\]

\begin{Shaded}
\begin{Highlighting}[]
\NormalTok{dni }\OtherTok{=} \FunctionTok{c}\NormalTok{(}\DecValTok{9}\NormalTok{, }\DecValTok{4}\NormalTok{, }\DecValTok{1}\NormalTok{, }\DecValTok{5}\NormalTok{, }\DecValTok{1}\NormalTok{, }\DecValTok{0}\NormalTok{, }\DecValTok{0}\NormalTok{, }\DecValTok{9}\NormalTok{)}
\end{Highlighting}
\end{Shaded}

El valor de \(dni^{2}\) es ~~~(81, 16, 1, 25, 1, 0, 0, 81)

El valor de \(\sqrt{dni}\) es ~~~(3, 2, 1, 2.236068, 1, 0, 0, 3)

La suma Acumulada del vector DNI es ~~~ 29

\hypertarget{punto-3---considerad-el-vector-de-las-letras-de-vuestro-nombre-y-apellido.-llamadlo-name.}{%
\paragraph{Punto 3 - Considerad el vector de las letras de vuestro
nombre y apellido. Llamadlo
name.}\label{punto-3---considerad-el-vector-de-las-letras-de-vuestro-nombre-y-apellido.-llamadlo-name.}}

\[name = (N,O,R,B,E,R,T,O,F,L,O,R,E,Z)\]

\begin{Shaded}
\begin{Highlighting}[]
\NormalTok{name }\OtherTok{=} \FunctionTok{c}\NormalTok{(}\StringTok{"N"}\NormalTok{,}\StringTok{"O"}\NormalTok{,}\StringTok{"R"}\NormalTok{,}\StringTok{"B"}\NormalTok{,}\StringTok{"E"}\NormalTok{,}\StringTok{"R"}\NormalTok{,}\StringTok{"T"}\NormalTok{,}\StringTok{"O"}\NormalTok{,}\StringTok{"F"}\NormalTok{,}\StringTok{"L"}\NormalTok{,}\StringTok{"O"}\NormalTok{,}\StringTok{"R"}\NormalTok{,}\StringTok{"E"}\NormalTok{,}\StringTok{"Z"}\NormalTok{)}
\end{Highlighting}
\end{Shaded}

El vector del nombre es ~ (N, O, R, B, E, R, T, O)

El vector del apellido es ~(F, L, O, R, E, Z)

El vector del apellido ordenado alfabeticamente es ~(E, F, L, O, R, Z)

\begin{Shaded}
\begin{Highlighting}[]
\FunctionTok{rbind}\NormalTok{(name[}\DecValTok{9}\SpecialCharTok{:}\DecValTok{14}\NormalTok{],name[}\DecValTok{9}\SpecialCharTok{:}\DecValTok{14}\NormalTok{],name[}\DecValTok{9}\SpecialCharTok{:}\DecValTok{14}\NormalTok{])}
\end{Highlighting}
\end{Shaded}

\begin{verbatim}
##      [,1] [,2] [,3] [,4] [,5] [,6]
## [1,] "F"  "L"  "O"  "R"  "E"  "Z" 
## [2,] "F"  "L"  "O"  "R"  "E"  "Z" 
## [3,] "F"  "L"  "O"  "R"  "E"  "Z"
\end{verbatim}

\[ Apellido = \begin{equation*}\begin{pmatrix} F & L & O & R & E & Z\\ F & L & O & R & E & Z\\ F & L & O & R & E & Z 
\end{pmatrix}\end{equation*}\]

\end{document}
